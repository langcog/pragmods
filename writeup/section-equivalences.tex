\section{Appendix A: Equivalences to other models}
\label{app:equivalences}

The RSA model we describe above has as special cases several previous models. In this section, we show these equivalences in slightly more detail.

\subsection{J\"ager (2013)}

The system described here is a restatement and generalization of the Iterated Best Response model described in J\"aeger's work \cite{jaegerinpress}. That work notates $S(C^T)$ as $\sigma$, and proposes an algorithm in which recursive applications of $L^*$ and $S^*$ are made until $X = L(S(X))$. 

\subsection{Frank \& Goodman (2012)}

In the original presentation of the RSA model, \citeA{frank2012} described a utility-theoretic derivation of a similar framework. They started with the idea that speakers choose messages relative to their utility with respect to the number of bits of information they would send to a simple, truth-functional listener; this formulation reduced to

\begin{equation}
P(w|r_S,C) = \frac{|w|^{-1}}{\displaystyle \sum_{w' \in W} {|w'|^{-1}}},
\end{equation}

where $|w|$ indicated the number of objects to which $w$ could refer. The associated listener probability was given by Bayesian inference from the speaker's likelihood and a prior term $P(r_S)$:

\begin{equation}
\label{eq:fg}
% P(r_S | w, C) \propto P(w | r_S, C) P(r_S).
P(r_S | w, C) 
= \frac{P(w | r_S, C) P(r_S)}{\displaystyle \sum_{r' \in C}{P(w | r', C) P(r')}} =
\frac{\frac{\displaystyle |w|^{-1}}{\displaystyle \sum_{w' \in W} {|w'|^{-1}}}P(r_S)}{\displaystyle \sum_{r' \in C}{\frac{|w|^{-1}}{\displaystyle \sum_{w' \in W} {|w'|^{-1}}}P(r')}}.
% \frac{|w|^{-1}}{\displaystyle \sum_{w' \in W} {|w'|^{-1}}}
\end{equation}

Working from our definitions, this formulation is equivalent to $L_B(S(L(C)))$. We can rewrite $L(C_{o,w})$ using the same notation as above, with $|w|$ as the number of objects to which a word refers. This allows us to write 

\begin{eqnarray*}
% L(C_{o,w}) = \frac{C_{o,w}}{\displaystyle\sum_{o' \in C} C_{o',w}} = |w|^{-1} \\
S(L(C_{o,w})) &=& \frac{|w|^{-1}}{\displaystyle \sum_{w' \in V(C)} |w|^{-1}} \mbox{, and} \\
L_B(S(L(C_{w,o}))) &=& \frac{ \frac{\displaystyle |w|^{-1}}{\displaystyle \sum_{w' \in V(C)} |w|^{-1}}P(o)}{\displaystyle\sum_{o' \in C}  \frac{|w|^{-1}}{\displaystyle \sum_{w' \in V(C)} |w|^{-1}}P(o')},
\end{eqnarray*}

which is equivalent to Equation \ref{eq:fg}.

\subsection{Other work}

Golland, Liang, and Klein \cite{golland2010} describe a similar system based on \cite{jaegerinpress}, in which they call $S(C^T)$ the ``reflex speaker'' and $S(L(C))$ the ``reasoned speaker.'' Benz \cite{benz2005b} describes a game-theoretic system that is similar to $S(L(C))$. 


%%% Local Variables: 
%%% TeX-master: "pragmods"
%%% End:
