
\section{General Methods}

\begin{figure}[t]
  \centering
  \includegraphics[width=4in]{figures/hatglasses.pdf}
  \caption{\label{fig:hg} An example stimulus item showing our canonical inferential context. The prompt in this case would be ``glasses,'' and the target item would be the face with glasses but not a hat. In all experiments reported here, position and identity of all features is randomized across displays; thus, the target is not always the face with glasses nor is it always positioned in the middle.}
\end{figure}

Our goal was to create a general method for measuring pragmatic inferences in simple reference games. Our taking-off point were previous studies by \citeA{frank2012} and \citeA{stiller2014}, in which simple feature-based displays allowed the measurement of pragmatic reasoning in grounded contexts. Following \citeA{stiller2014}, we created a set


\subsection{Samples}

We converged quickly on a general standard of 50 participants per cell, based on the tradeoff between cost and the desired precision of the estimates on our measurements. See SI for simulations showing estimated confidence intervals on a forced-choice measure. 

All of the experiments described here were run on Amazon's Mechanical Turk crowdsourcing service, between Fall 2013 and Spring of 2015. 

\subsection{Stimuli}

We created a set of ``base'' features

\subsection{Procedure}

Participants viewed the experiment within a browser window. The first screen of the experiment presented a basic description of the paradigm and asked for informed consent. The second screen of the experiment presented the interlocutor, Bob (a cartoon picture of a man), and noted that he liked to do activities with the base item (e.g., visit friends or sail boats). In experiments with familiarization stimuli (e.g., \Exptref{exp:prior-fam})


% Minor stuff:

% Prelims measures has four items

% Size and sequence have no manipulation check 

%%% Local Variables: 
%%% TeX-master: "pragmods"
%%% End:
