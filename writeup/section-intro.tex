\section{Introduction}

Human communication is astonishingly flexible and efficient. Between two people who know each other well, a single word, look, or gesture often conveys volumes \cite{grice1975,sperber1986,clark1996}. Even two people who barely know one another can rapidly leverage their shared vocabulary to communicate about novel situations \cite{brennan1993}.  This efficiency is astonishing, given the vagueness of language. 

As a solution to this problem, \citeA{grice1975} proposed that speakers and listeners reason cooperatively together to 


\subsection{The current work}

The current work has two interlocking and primary goals. The first goal is empirical: via a sequence of large-scale, web-based experiments, we provide strong evidence that one-shot reference games provide a flexible and precise tool for studying quantitative patterns of human behavior. The second goal is formal: we provide a more comprehensive presentation of the ``rational speech act'' (RSA) model of pragmatic reasoning and show that it predicts the patterns of performance we observed in our experiments. Taken together, this body of work suggests that the RSA model provides a powerful framework for studying human pragmatic reasoning in quantitative detail. 

Our experiments are listed in Table \ref{tab:expts}; they are broken into six sequences. The first of these (``Preliminaries,'' Section \ref{sec:prelims}) explores our primary experimental paradigm, one-shot reference games, providing experimental evidence that minor design choices do not account for the pattern of data we observe. The second section (``Priors,'' Section \ref{sec:priors}) tests the role of prior expectations in reference game behavior. The third (``Levels,'' Section \ref{sec:levels}) tests the level of recursion that speakers reason to and provides further data on the relationship between prior and posterior measurements. The fourth (``Sequences,'' Section \ref{sec:seqs}) takes a first step towards exploring sequential effects in pragmatic reasoning and whether these can scaffold more deeply referential expressions. The fifth and final section (``Production,'' Section \ref{sec:prod}) discusses speaker behavior in reference games and shows that it is sensitive to the costs of language production. 

Our plan is as follows. We first present formal details of the RSA model. We next move to our experimental presentation. We then end by discussing the fit of RSA models to our data, focusing primarily on data from Sections \ref{sec:priors} and \ref{sec:levels}. To conclude, we discuss limitations and future directions for RSA modeling, as well implications of this work for the study of pragmatics more broadly. 

%%% Local Variables: 
%%% TeX-master: "pragmods"
%%% End:

