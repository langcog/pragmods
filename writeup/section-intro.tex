\section{Introduction}

Human communication is astonishingly flexible and efficient. Between two people who know each other well, a single word, look, or gesture can convey volumes \cite{sperber1986,clark1996}. Even two people who barely know one another can rapidly leverage their shared vocabulary to create new, economical ways of communicating about novel situations \cite{brennan1996,clark1991}. These successes are all the more impressive considering the systematic issues of vagueness and ambiguity in natural language \cite{keefe1997,wasow2005}. How do we use such slippery means to produce such concrete results? 

\citeA{grice1975} provided a critical insight into this problem: listeners could reason about speakers' goals in choosing a particular form for their communication. By assuming that speakers were attempting to be cooperative in their communicative, listeners could then infer the most likely intended goal that would have given rise to that form, given the current circumstances. This proposal allowed the separation of semantic content---those aspects of meaning that are invariant across contexts---from pragmatic content---those aspects of meaning that rely on contextual inferences about speaker intentions. 

Although many subsequent analysts have attempted to refine Grice's original proposal or even to reject substantial elements, the semantics/pragmatics distinction has been critical for progress in understanding language comprehension. Nevertheless, because of the difficulties of formalizing notions like context and social goal inference, pragmatics has often remained a ``wastebasket'' for phenomena that are difficult to explain under formal theories of syntax or semantics \cite{bar-hillel1971}. This dismissal of pragmatics is regrettable because of the importance and richness of pragmatic phenomena. In some substantial sense, what linguists call ``pragmatics'' is what language users experience as the use of language in their everyday interpersonal communication---the reality of how context alters interpretation. Thus, an important goal for research in linguistics and the psychology of language is the development of formal tools for understanding pragmatic inferences and bringing them under experimental control. 

In this article, we report on a formal framework for understanding pragmatics, known as the ``rational speech act'' (RSA) framework \cite<originally introduced in>{frank2012,goodman2013}. This framework builds on Grice's initial insight and combines it with tools from Bayesian cognitive modeling \cite{tenenbaum2011}. The result is a set of tools for a quantitative understanding of the kind of social goal inference that Grice initially described. In the remainder of this introduction, we provide a relatively brief survey of qualitative frameworks for pragmatic reasoning, and then discuss quantitative progress in quantitative models of pragmatics. We then summarize the plan for the rest of the article.

\subsection{Qualitative models of pragmatic reasoning}

Grice

Horn

Levinson

Sperber and Wilson

Chierchia et al. 

\subsection{Quantitative models of pragmatic reasoning}

Early work, hobbes, etc. 

referring expression generation

percy liang

Game-theoretic approaches - parikh, van rooj, Franke, Jaeger

RSA models - antecedents in social reasoning models \cite{baker2009}

\subsection{The current work}

The current work has two interlocking goals. The first goal is empirical: via a sequence of large-scale, web-based experiments, we provide strong evidence that one-shot reference games provide a flexible and precise tool for studying quantitative patterns of human behavior. The second goal is formal: we provide a more comprehensive presentation of ``rational speech act'' (RSA) models of pragmatic reasoning and show that this framework captures many of the patterns of performance we observed in our experiments. Taken together, this body of work suggests that the RSA model provides a powerful set of tools for studying human pragmatic reasoning in quantitative detail. 

Our experiments are listed in Table \ref{tab:expts}; they are broken into three sequences. The first of these (``Preliminaries,'' Section \ref{sec:prelims}) explores our primary experimental paradigm, one-shot reference games, providing experimental evidence that minor design choices do not account for the pattern of data we observe. The second section (``Priors,'' Section \ref{sec:prior}) tests the role of prior expectations in reference game behavior. The third (``Levels,'' Section \ref{sec:levels}) tests the level of recursion that speakers reason to and provides further data on the relationship between prior and posterior measurements. 
% The fourth (``Sequences,'' Section \ref{sec:seqs}) takes a first step towards exploring sequential effects in pragmatic reasoning and whether these can scaffold more deeply referential expressions. The fifth and final section (``Production,'' Section \ref{sec:prod}) discusses speaker behavior in reference games and shows that it is sensitive to the costs of language production. 

Our plan is as follows. We first present formal details of the RSA model. We next move to our experimental presentation. We then examine the fit of RSA models to our data. To conclude, we discuss limitations and future directions for RSA modeling, as well implications of this work for the study of pragmatics more broadly. 

%%% Local Variables: 
%%% TeX-master: "pragmods"
%%% End:

